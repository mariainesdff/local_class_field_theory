%%
%% This is file `sample-sigplan.tex',
%% generated with the docstrip utility.
%%
%% The original source files were:
%%
%% samples.dtx  (with options: `sigplan')
%% 
%% IMPORTANT NOTICE:
%% 
%% For the copyright see the source file.
%% 
%% Any modified versions of this file must be renamed
%% with new filenames distinct from sample-sigplan.tex.
%% 
%% For distribution of the original source see the terms
%% for copying and modification in the file samples.dtx.
%% 
%% This generated file may be distributed as long as the
%% original source files, as listed above, are part of the
%% same distribution. (The sources need not necessarily be
%% in the same archive or directory.)
%%
%%
%% Commands for TeXCount
%TC:macro \cite [option:text,text]
%TC:macro \citep [option:text,text]
%TC:macro \citet [option:text,text]
%TC:envir table 0 1
%TC:envir table* 0 1
%TC:envir tabular [ignore] word
%TC:envir displaymath 0 word
%TC:envir math 0 word
%TC:envir comment 0 0
%%
%%
%% The first command in your LaTeX source must be the \documentclass
%% command.
%%
%% For submission and review of your manuscript please change the
%% command to \documentclass[manuscript, screen, review]{acmart}.
%%
%% When submitting camera ready or to TAPS, please change the command
%% to \documentclass[sigconf]{acmart} or whichever template is required
%% for your publication.
%%
%%
%\documentclass[sigplan,screen]{acmart}
\documentclass[sigplan,10pt,anonymous,review]{acmart}\settopmatter{printfolios=true,printccs=false,printacmref=false}

\usepackage[T1]{fontenc}

%%%%%%%%%%%%%%%%%%%%%%%%%%%%%%%%%%%%%%%%%%%
% Math commands
%%%%%%%%%%%%%%%%%%%%%%%%%%%%%%%%%%%%%%%%%%%
\newcommand*{\ZZ}{\mathbb{Z}}

%end of preamble

%%
%% \BibTeX command to typeset BibTeX logo in the docs
\AtBeginDocument{%
  \providecommand\BibTeX{{%
    Bib\TeX}}}

%% Rights management information.  This information is sent to you
%% when you complete the rights form.  These commands have SAMPLE
%% values in them; it is your responsibility as an author to replace
%% the commands and values with those provided to you when you
%% complete the rights form.
\setcopyright{acmcopyright}
\copyrightyear{2023}
\acmYear{2024}
\acmDOI{XXXXXXX.XXXXXXX}

%% These commands are for a PROCEEDINGS abstract or paper.
\acmConference[CPP 2024]{Certified Programs and Proofs 2024}{January 15--16,
  2024}{London, UK}
%%
%%  Uncomment \acmBooktitle if the title of the proceedings is different
%%  from ``Proceedings of ...''!
%%
%%\acmBooktitle{Woodstock '18: ACM Symposium on Neural Gaze Detection,
%%  June 03--05, 2018, Woodstock, NY}
\acmPrice{15.00}
\acmISBN{978-1-4503-XXXX-X/18/06}


%%
%% Submission ID.
%% Use this when submitting an article to a sponsored event. You'll
%% receive a unique submission ID from the organizers
%% of the event, and this ID should be used as the parameter to this command.
%%\acmSubmissionID{123-A56-BU3}

%%
%% For managing citations, it is recommended to use bibliography
%% files in BibTeX format.
%%
%% You can then either use BibTeX with the ACM-Reference-Format style,
%% or BibLaTeX with the acmnumeric or acmauthoryear sytles, that include
%% support for advanced citation of software artefact from the
%% biblatex-software package, also separately available on CTAN.
%%
%% Look at the sample-*-biblatex.tex files for templates showcasing
%% the biblatex styles.
%%

%%
%% The majority of ACM publications use numbered citations and
%% references.  The command \citestyle{authoryear} switches to the
%% "author year" style.
%%
%% If you are preparing content for an event
%% sponsored by ACM SIGGRAPH, you must use the "author year" style of
%% citations and references.
%% Uncommenting
%% the next command will enable that style.
%%\citestyle{acmauthoryear}


%%
%% end of the preamble, start of the body of the document source.
\begin{document}

%%
%% The "title" command has an optional parameter,
%% allowing the author to define a "short title" to be used in page headers.
\title[A Formalization of Complete DVRs and Local Fields]{A Formalization of Complete Discrete Valuation Rings and Local Fields}

%%
%% The "author" command and its associated commands are used to define
%% the authors and their affiliations.
%% Of note is the shared affiliation of the first two authors, and the
%% "authornote" and "authornotemark" commands
%% used to denote shared contribution to the research.
\iffalse
\author{Ben Trovato}
\authornote{Both authors contributed equally to this research.}
\email{trovato@corporation.com}
\orcid{1234-5678-9012}
\author{G.K.M. Tobin}
\authornotemark[1]
\email{webmaster@marysville-ohio.com}
\affiliation{%
  \institution{Institute for Clarity in Documentation}
  \streetaddress{P.O. Box 1212}
  \city{Dublin}
  \state{Ohio}
  \country{USA}
  \postcode{43017-6221}
}\fi

% \author[M.~I.~de Frutos Fernández]{María Inés de Frutos Fernández}
\author{María Inés de Frutos Fernández}
\email{maria.defrutos@uam.es}
\orcid{0000-0002-5085-7446}
\affiliation{%
  \institution{Universidad Aut\'onoma de Madrid}
  \streetaddress{Ciudad Universitaria de Cantoblanco}
   \postcode{28049}
  \city{Madrid}
  \country{Spain}
}

%\author[F.~A.~E.~Nuccio]{Filippo Alberto Edoardo Nuccio Mortarino Majno di Capriglio}
\author{Filippo Alberto Edoardo Nuccio Mortarino Majno di Capriglio}
\email{}
\orcid{}
\affiliation{%
	\institution{}
	\city{}
	\country{France}
}


%%
%% By default, the full list of authors will be used in the page
%% headers. Often, this list is too long, and will overlap
%% other information printed in the page headers. This command allows
%% the author to define a more concise list
%% of authors' names for this purpose.
\renewcommand{\shortauthors}{M.~I.~de Frutos Fernández and F.~A.~E.~Nuccio }

%%
%% The abstract is a short summary of the work to be presented in the
%% article.
\begin{abstract}
  Local fields, and fields complete with respect to a discrete valuation, are essential objects in commutative algebra, with applications to number theory and algebraic geometry. We formalize in Lean the basic theory of discretely valued fields. In particular, we prove that the unit ball with respect to a discrete valuation on a field is a discrete valuation ring, and conversely, that the adic
  valuation on the field of fractions of a discrete valuation ring is discrete. We define finite extensions of valuations and of discrete valuation
  rings, and prove some global-to-local results. 
  
  Building on this general theory, we formalize the abstract definition of local fields and some of their fundamental properties. As an application, we show that finite extensions of the field $\QQ_p$ of $p$-adic numbers and of the field $\laurentseries$ of Laurent series over a finite field are local fields. 
  
\end{abstract}

%%
%% The code below is generated by the tool at http://dl.acm.org/ccs.cfm.
%% Please copy and paste the code instead of the example below. (DONE)
%%
\begin{CCSXML}
	<ccs2012>
	<concept>
	<concept_id>10003752.10003790.10002990</concept_id>
	<concept_desc>Theory of computation~Logic and verification</concept_desc>
	<concept_significance>500</concept_significance>
	</concept>
	<concept>
	<concept_id>10003752.10003790.10003792</concept_id>
	<concept_desc>Theory of computation~Proof theory</concept_desc>
	<concept_significance>500</concept_significance>
	</concept>
	</ccs2012>
\end{CCSXML}

\ccsdesc[500]{Theory of computation~Logic and verification}
\ccsdesc[500]{Theory of computation~Proof theory}


%%
%% Keywords. The author(s) should pick words that accurately describe
%% the work being presented. Separate the keywords with commas.
\keywords{formal mathematics, Lean, mathlib, algebraic number theory, local fields, discrete valuation rings.}


%\received{20 February 2007}
%\received[revised]{12 March 2009}
%\received[accepted]{5 June 2009}

%%
%% This command processes the author and affiliation and title
%% information and builds the first part of the formatted document.
\maketitle

\section{Introduction}
\begin{itemize}
\item Say all rings are commutative and unitary.
\end{itemize}
\section {Discrete Valuation Rings}
\mi{
\subsection{Valuations}
\begin{itemize}
	\item Properties of valuations valued in $\ZZ_{m0}$, and the \texttt{multiplicative} and \texttt{with\_zero} classes, and clashes with "usual" intuition. Keep in mind that \texttt{with\_zero multiplicative} $\ZZ$ is "the same" as $q^{-\ZZ}\cup \{0\}$.
	\item Difference between \texttt{valued} (which is a class, and carries a topology/uniformity) and \texttt{valuation} that is just a function.
	\item We put \texttt{valued} instances on fields but "never" on rings that are not fields.
	\item There is something called \texttt{valuation\_ring} that has (almost) nothing to do; but \texttt{valuation\_subring} has.
	\item Comment about the choice of \texttt{.integers} vs. \texttt{.valuation\_subring}
\end{itemize}
}
\fil{
\subsection{Definition of Discrete Valuation Rings}
\begin{itemize}
	\item True Math definition and the \texttt{mathlib\_tfae} (in particular, local and PID)
	\item \texttt{is\_discrete} : observe that if the value group were $2\ZZ$ this would still be mathematically "discrete", but we make the choice of surjectivity in coherence with Serre/so that the notion of equivalence collapses to equality: there is a unique normalized one.
	\item \texttt{uniformizer}'s and equivalence of their existence with discreteness
	\mi{\item The unit ball of a field with a discrete valuation (be it valued or not) is a DVR; and the adic valuation (ref to María Inés' paper) on the field of fractions of a DVR \texttt{is\_discrete}}.
\end{itemize}
}
\subsection{Complete Discrete Valuation Rings and their Extensions}
Let $R$ be a ring, as above assumed commutative and unitary. We begin by recalling the following
\begin{theorem}[{\cite{Bou07, Chap. VII, §2, no. 2, Théorème 1]}}]
The following properties are equivalent:
%\begin{listResults}
\end{theorem}
\begin{itemize}
	\item The completion of a discretely valued field is again discretely valued (and so its ring of integers is a DVR by the above)
	\fil{\item The equality of the two valuations (the adic one coming from the max'l ideal, since DVR implies local) and the uniform completion of the valuation we begun with
	\item When viewing $K_v$ as the completion of $K$, its \texttt{valued} instance comes from the completion of the valuation on $K$, and this is of course different from the \texttt{valued} instance on the fraction field of $R_v$, itself isomorphic to $K_v$, that instead comes from the \path{discrete_valuation_ring} instance on $R_v$.
	\item Still, the isomorphism is known and one can compare the two valuations on $K_v$ which is \emph{a fraction field}; this is done in \path{v_1=v_2}.}
	\mi{\item Recall (María Inés' paper): there is a norm associated to a valuation; and if the field below is complete and the extension is finite, then this norm is unique (extending the one below).
	\item Speak about the work to define the normalized \texttt{discrete} valuation "above"
	\item [MAYBE] Mention the lemma \texttt{w\_le\_one\_iff\_disc\_norm\_extension\_le\_one}
	\item The proof (from previous project) and instance that a finite extension of a complete field is again complete
	\item The lemma \path{integral_closure_eq_integer} and the theorem \path{dvr_of_finite_extension}
	\item The extension of the valuation is not a valued instance because this would create a diamond for every valued field K. But we do get a DVR instance on the integral closure of the integers.}
\end{itemize}
\section{Local Fields}
Give our definition and the claims that finite extensions of $\laurentseries$ and of $\QQ_p$ are local fields.\\
The valued instance on a mixed/equal characteristic local field causes a diamond on $\QQ_p$ or
$\laurentseries$.
\subsection{Equal characteristic}
\begin{itemize}
	\item Unlike the $p$-adic, the completion of the ring of rational functions had not yet been defined in mathlib.
	\item Put both a valuation and a norm on $\laurentseries$
	\item iso between Laurent series and the completion
	\item Definition of Eq char local field
	\item Our definition of ring of integers is the integral closure of $\powerseries$ in $K$, mainly to adapt to what has been done for number fields.
	\item The definition \path{ring_of_integers_algebra}
	\item $\laurentseries$ is an eq. char. local ring [(and also $\laurentseries[{\FF[q]}]$?) \fil{NOT DONE: AVOID?}]
	
\end{itemize}
\subsection{Mixed characteristic}
\begin{itemize}
	\item Definition of Mixed char local field
	\item Our definition of ring of integers is the integral closure of $\ZZ_p$ in $K$, mainly to adapt to what has been done for number fields.
	\item The definition \path{ring_of_integers_algebra}, "equal" to the one for eq. char. local fields
	\item The proof that a $\QQ_p$ is a mixed char local field.
	\item Discussion about the two $\QQ_p$'s from a mathematical point of view.
	\item Say that most of the techniques used for the iso about Laurent series adapted to \path{padic_ring_equiv} IF TRUE
	\item A proof that the ring of integers is a DVR (once $\ZZ_p$ becomes the right $\ZZ_p$ --- so, not only a DVR, which it is, but the \path{valuation_subring} of $\QQ_p$ --- this will follow from the general results about DVR's).
\end{itemize}


\section{Discussion}
\begin{itemize}
	\item Localizations of number fields are local fields
	\item Future work (not too much, but towards LCFT)
	\item Related works
\end{itemize}
\subsection{Implementation Details}
\begin{itemize}
	\item bundled vs unbundled \texttt{uniformizer}'s and the choice of having them in K or in R or in R.int\dots
	\item insert here the "double" $Q_p$ in particular the fact we found a \path{metric} instance on $\QQ$.
	\item the loop forcing the instance \texttt{discretely\_normed\_field} to be only a local one
	\item [MAY BE] Discuss working in $\ZZ_{m0}$ vs in $\RR$ for the DVR\_extension
	\item The instance of \path{valued} at the end of basic is defined on THE fraction ring and not on A fraction field because otherwise this would get into a diamond. Look at the examples in \path{extension}. Observe that \path{height_one_spectrum.valued_adic_completion K v} takes as argument $K$ (that is A fraction ring of $R$), whereas \path{discrete_valuation.with_zero.valued} takes as argument a DVR (in this case, $R_v$). Since the completion at $v$ of the fraction ring $frac R$ is not equal to $K_v$, this leads to no problem.
\end{itemize}

%%
%% The acknowledgments section is defined using the "acks" environment
%% (and NOT an unnumbered section). This ensures the proper
%% identification of the section in the article metadata, and the
%% consistent spelling of the heading.
%TODO We should add the funding ack. here (this is commented out by the `anonymous` option).
\begin{acks}
\mi{}
\end{acks}

%%
%% The next two lines define the bibliography style to be used, and
%% the bibliography file.
\bibliographystyle{ACM-Reference-Format}
\bibliography{CPP_biblio}


%%
%% If your work has an appendix, this is the place to put it.
%\appendix


\end{document}
\endinput
%%
%% End of file `sample-sigplan.tex'.
