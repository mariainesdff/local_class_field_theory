%%
%% This is file `sample-sigplan.tex',
%% generated with the docstrip utility.
%%
%% The original source files were:
%%
%% samples.dtx  (with options: `sigplan')
%% 
%% IMPORTANT NOTICE:
%% 
%% For the copyright see the source file.
%% 
%% Any modified versions of this file must be renamed
%% with new filenames distinct from sample-sigplan.tex.
%% 
%% For distribution of the original source see the terms
%% for copying and modification in the file samples.dtx.
%% 
%% This generated file may be distributed as long as the
%% original source files, as listed above, are part of the
%% same distribution. (The sources need not necessarily be
%% in the same archive or directory.)
%%
%%
%% Commands for TeXCount
%TC:macro \cite [option:text,text]
%TC:macro \citep [option:text,text]
%TC:macro \citet [option:text,text]
%TC:envir table 0 1
%TC:envir table* 0 1
%TC:envir tabular [ignore] word
%TC:envir displaymath 0 word
%TC:envir math 0 word
%TC:envir comment 0 0
%%
%%
%% The first command in your LaTeX source must be the \documentclass
%% command.
%%
%% For submission and review of your manuscript please change the
%% command to \documentclass[manuscript, screen, review]{acmart}.
%%
%% When submitting camera ready or to TAPS, please change the command
%% to \documentclass[sigconf]{acmart} or whichever template is required
%% for your publication.
%%
%%
%\documentclass[sigplan,screen]{acmart}
%\documentclass[sigplan,10pt,anonymous,screen]{acmart}
\documentclass[sigplan,10pt,anonymous,review]{acmart}
\settopmatter{printfolios=true,printccs=false,printacmref=false}

\usepackage[T1]{fontenc}
% \PassOptionsToPackage{rgb, dvipsnames}{xcolor}
\definecolor{Burgundy}{RGB}{128,0,32}
\usepackage{enumitem}
\usepackage{amsfonts, amsthm, amsmath}
\usepackage{soul}
\usepackage[english]{babel}
\hyphenation{non-archi-me-de-an}
% \usepackage{dutchcal}
\usepackage{mathrsfs}
\usepackage{fontawesome} % external link icon
\usepackage{xpunctuate}
\usepackage{hyperref}
\usepackage{microtype}
 \usepackage{balance}
%\hypersetup{%
%	colorlinks=true,
%	linkcolor=blue,
%	filecolor=magenta,      
%	urlcolor=blue,
%}
%%%%%%%%%%%%%%%%%%%%%%%%%%%%%%%%%%%%%%%%%%%
% Typography
%%%%%%%%%%%%%%%%%%%%%%%%%%%%%%%%%%%%%%%%%%%
 \newcommand{\fil}[1]{{\color{blue} #1}}
 \newcommand{\mi}[1]{{\color{purple} #1}}
%\DeclareRobustCommand{\extlink}[1]{\href{{#1}}{\ensuremath{{}^\text{\faExternalLink}}}}
\newcommand{\extlink}{~\ensuremath{{}^\text{\faExternalLink}}}
\newcommand*{\ibid}{\emph{ibid}\xperiod}
%%%%%%%%%%%%%%%%%%%%%%%%%
%%Lean-related typography
%%%%%%%%%%%%%%%%%%%%%%
\usepackage{listings}
\def\lstlanguagefiles{lstleanCPP.sty}
\lstset{language=lean}
%\definecolor{keywordcolor}{rgb}{0.7, 0.1, 0.1}   % red
%\definecolor{commentcolor}{rgb}{0.4, 0.4, 0.4}   % grey
%\definecolor{symbolcolor}{rgb}{0.4, 0.4, 0.4}    % grey
\renewcommand{\lstlistingname}{Code excerpt}
\newcommand{\LClistingname}{\MakeLowercase{\lstlistingname}}
\newcommand{\code}[1]{\lstinline{#1}}
\newcommand{\mcode}[1]{\mathtt{\ #1\ }}
\newcommand*{\lean}[1][3]{Lean~{#1}\xspace}
\newcommand*{\mathlib}{\texttt{mathlib}\xspace}
% \newcommand{\pfomit}{$ - - QED$}
%%%%%%%%%%%%%%%%%%%%%%%%%%%%%%%%%%%%%%%%%%%
%%%Theorems
%%%%%%%%%%%%%%%%%%%%%%%%%%%%%%%%%%%%%%%%%%%
\theoremstyle{plain}
\newtheorem{theorem}{Theorem}[section]
\newtheorem{proposition}[theorem]{Proposition}
\newtheorem{corollary}[theorem]{Corollary}
\theoremstyle{definition}
\newtheorem{definition}[theorem]{Definition}
\newtheorem{notation}[theorem]{Notation}
%
\theoremstyle{remark}
\newtheorem{remark}[theorem]{Remark}
%%%%%%%%%%%%%%%%%%%%%%%%%%%%%%%%%%%%%%%%%%%
%%%%%Lists
%%%%%%%%%%%%%%%%%%%%%%%%%%%%%%%%%%%%%%%%%%%
\newlist{listResults}{enumerate}{1}
\setlist[listResults]{label={\arabic*}.}
\newlist{sublistResults}{enumerate}{1}
\setlist[sublistResults]{label={\arabic{listResultsi}}-\,\alph*\textup{)}}
\newlist{listConditions}{enumerate}{1}
\setlist[listConditions]{label=\textup{({\Alph*})}}
\newlist{listDef}{enumerate}{1}
\setlist[listDef]{label={\roman*}\textup{)}}
\newlist{listHyp}{enumerate}{1}
%%%%%%%%%%%%%%%%%%%%%%%%%%%%%%%%%%%%%%%%%%%
% Math commands
%%%%%%%%%%%%%%%%%%%%%%%%%%%%%%%%%%%%%%%%%%%
%%Fields
\newcommand*{\NN}{\mathbb{N}}
\newcommand*{\ZZ}{\mathbb{Z}}
\newcommand*{\wzmZZ}{\ZZ_{\mathtt{m}0}} %% With-zero multiplicative version of \ZZ
\newcommand*{\wzmNN}{\NN_{\mathtt{m}0}} %% With-zero multiplicative version of \NN
\newcommand*{\RR}{\mathbb{R}}
\newcommand*{\QQ}{\mathbb{Q}}
\newcommand*{\CC}{\mathbb{C}}
\newcommand*{\FF}[1][p]{\mathbb{F}_{#1}}
\newcommand*{\m}{\mathfrak{m}}
\newcommand*{\av}[1][\empty]{a_{#1}}
\DeclareMathOperator{\Frac}{Frac} %the fraction field
%%Algebraic
\newcommand*{\laurentseries}[1][{\FF}]{\ensuremath{#1(\!(X)\!)}} %Laurent series
\newcommand*{\puiseuxseries}[1][{R}]{\ensuremath{#1\{\!\{X\}\!\}}} %Laurent series
\newcommand*{\compl}[2][\empty]{\ensuremath{\widehat{#2}_{#1}}} %General completion
\newcommand*{\FpXCompl}[1][{\FF}]{\compl{#1(X)}}  %Adic completion of Rational functions
\newcommand*{\FpXComplInt}[1][\FF]{\ball{\bigl(\FpXCompl[#1]\bigr)}}  %Adic completion of polynomials
\newcommand*{\padicCompl}{\compl[(p)]{\QQ}}  %Adic completion of Rational numbers
\newcommand*{\padicComplInt}{\ball{\bigl(\padicCompl\bigr)}}  %Adic completion of the integers
\newcommand*{\powerseries}[1][{\FF}]{#1{[\![X]\!]}} %Power series
\newcommand*{\maxid}[1][R]{\mathfrak{m}_{#1}} %Maximal ideal (of R)
\newcommand{\unit}[1]{{#1}^\times} % units
%% Arithmetic
\newcommand*{\rint}[1]{\mathcal{O}_{#1}} % ring of integers
%% Valuations
\newcommand*{\normalised}[1][v]{{#1}_{0}} %The normalised valuation attached to v
\newcommand*{\ball}[1]{{#1}^\circ} % ring of integers
%%General set-theoretic
\newcommand*{\filter}{\mathscr{F}}
\newcommand*{\princfilter}{\boldsymbol{\mathcal{P}}}
\newcommand*{\nhds}{\boldsymbol{\mathcal{N}}}
\DeclareMathOperator{\image}{Im} %image of a map
\newcommand*{\longtwoheadrightarrow}{\joinrel\relbar\joinrel\twoheadrightarrow}%Long surjective arrow

%end of preamble

%%
%% \BibTeX command to typeset BibTeX logo in the docs
\AtBeginDocument{%
  \providecommand\BibTeX{{%
    Bib\TeX}}}

%% Rights management information.  This information is sent to you
%% when you complete the rights form.  These commands have SAMPLE
%% values in them; it is your responsibility as an author to replace
%% the commands and values with those provided to you when you
%% complete the rights form.
\setcopyright{acmcopyright}
\copyrightyear{2023}
\acmYear{2024}
\acmDOI{XXXXXXX.XXXXXXX}

%% These commands are for a PROCEEDINGS abstract or paper.
\acmConference[CPP 2024]{Certified Programs and Proofs 2024}{January 15--16,
  2024}{London, UK}
%%
%%  Uncomment \acmBooktitle if the title of the proceedings is different
%%  from ``Proceedings of ...''!
%%
%%\acmBooktitle{Woodstock '18: ACM Symposium on Neural Gaze Detection,
%%  June 03--05, 2018, Woodstock, NY}
\acmPrice{15.00}
\acmISBN{978-1-4503-XXXX-X/18/06}


%%
%% Submission ID.
%% Use this when submitting an article to a sponsored event. You'll
%% receive a unique submission ID from the organizers
%% of the event, and this ID should be used as the parameter to this command.
%%\acmSubmissionID{123-A56-BU3}

%%
%% For managing citations, it is recommended to use bibliography
%% files in BibTeX format.
%%
%% You can then either use BibTeX with the ACM-Reference-Format style,
%% or BibLaTeX with the acmnumeric or acmauthoryear sytles, that include
%% support for advanced citation of software artefact from the
%% biblatex-software package, also separately available on CTAN.
%%
%% Look at the sample-*-biblatex.tex files for templates showcasing
%% the biblatex styles.
%%

%%
%% The majority of ACM publications use numbered citations and
%% references.  The command \citestyle{authoryear} switches to the
%% "author year" style.
%%
%% If you are preparing content for an event
%% sponsored by ACM SIGGRAPH, you must use the "author year" style of
%% citations and references.
%% Uncommenting
%% the next command will enable that style.
%%\citestyle{acmauthoryear}


%%
%% end of the preamble, start of the body of the document source.
\begin{document}

%%
%% The "title" command has an optional parameter,
%% allowing the author to define a "short title" to be used in page headers.
\title[A Formalization of Complete DVRs and Local Fields]{A Formalization of Complete Discrete Valuation Rings and Local Fields}

%%
%% The "author" command and its associated commands are used to define
%% the authors and their affiliations.
%% Of note is the shared affiliation of the first two authors, and the
%% "authornote" and "authornotemark" commands
%% used to denote shared contribution to the research.
\iffalse
\author{Ben Trovato}
\authornote{Both authors contributed equally to this research.}
\email{trovato@corporation.com}
\orcid{1234-5678-9012}
\author{G.K.M. Tobin}
\authornotemark[1]
\email{webmaster@marysville-ohio.com}
\affiliation{%
  \institution{Institute for Clarity in Documentation}
  \streetaddress{P.O. Box 1212}
  \city{Dublin}
  \state{Ohio}
  \country{USA}
  \postcode{43017-6221}
}\fi

% \author[M.~I.~de Frutos Fernández]{María Inés de Frutos Fernández}
\author{María Inés de Frutos Fernández}
\email{maria.defrutos@uam.es}
\orcid{0000-0002-5085-7446}
\affiliation{%
  \institution{Universidad Aut\'onoma de Madrid}
  \streetaddress{Ciudad Universitaria de Cantoblanco}
   \postcode{28049}
  \city{Madrid}
  \country{Spain}
}

%\author[F.~A.~E.~Nuccio]{Filippo Alberto Edoardo Nuccio Mortarino Majno di Capriglio}
\author{Filippo Alberto Edoardo Nuccio Mortarino Majno di Capriglio}
\email{}
\orcid{}
\affiliation{%
	\institution{}
	\city{}
	\country{France}
}


%%
%% By default, the full list of authors will be used in the page
%% headers. Often, this list is too long, and will overlap
%% other information printed in the page headers. This command allows
%% the author to define a more concise list
%% of authors' names for this purpose.
\renewcommand{\shortauthors}{M.~I.~de Frutos Fernández and F.~A.~E.~Nuccio }

%%
%% The abstract is a short summary of the work to be presented in the
%% article.
\begin{abstract}
  Local fields, and fields complete with respect to a discrete valuation, are essential objects in commutative algebra, with applications to number theory and algebraic geometry. We formalize in Lean the basic theory of discretely valued fields. In particular, we prove that the unit ball with respect to a discrete valuation on a field is a discrete valuation ring and, conversely, that the adic
  valuation on the field of fractions of a discrete valuation ring is discrete. We define finite extensions of valuations and of discrete valuation
  rings, and prove some global-to-local results. 
  
  Building on this general theory, we formalize the abstract definition and some fundamental properties of local fields. As an application, we show that finite extensions of the field $\QQ_p$ of $p$-adic numbers and of the field $\laurentseries$ of Laurent series over $\FF$ are local fields. 
  
\end{abstract}

%%
%% The code below is generated by the tool at http://dl.acm.org/ccs.cfm.
%% Please copy and paste the code instead of the example below. (DONE)
%%
\begin{CCSXML}
	<ccs2012>
	<concept>
	<concept_id>10003752.10003790.10002990</concept_id>
	<concept_desc>Theory of computation~Logic and verification</concept_desc>
	<concept_significance>500</concept_significance>
	</concept>
	<concept>
	<concept_id>10003752.10003790.10003792</concept_id>
	<concept_desc>Theory of computation~Proof theory</concept_desc>
	<concept_significance>500</concept_significance>
	</concept>
	</ccs2012>
\end{CCSXML}

\ccsdesc[500]{Theory of computation~Logic and verification}
\ccsdesc[500]{Theory of computation~Proof theory}


%%
%% Keywords. The author(s) should pick words that accurately describe
%% the work being presented. Separate the keywords with commas.
\keywords{formal mathematics, Lean, mathlib, algebraic number theory, local fields, discrete valuation rings.}


%\received{20 February 2007}
%\received[revised]{12 March 2009}
%\received[accepted]{5 June 2009}

%%
%% This command processes the author and affiliation and title
%% information and builds the first part of the formatted document.
\maketitle

\section{Introduction}
\begin{itemize}
\item Say all rings are commutative and unitary.
\item Remember to create the graph
\item Acknowledge de Frutos-Fernández and Yaël Dillies for allowing us to use their work.
\item Add  \href{https://github.com/mariainesdff/local_class_field_theory}{link to repository}. \mi{Links are rendered in black with the "review" option, which means they will probably be missed. In the sphere eversion paper they have a symbol next to each link; do you know how to achieve this? Otherwise we could ask them.}
\end{itemize}
\section {Discrete Valuation Rings}\label{section:dvr}

\subsection{Valuations}\label{subsection:valuations}
Let $R$ be a ring. A \textit{valuation} $v$ on $R$ is a map $v : R \to \Gamma_0$  to a linearly ordered commutative monoid with zero $\Gamma_0$ such that
\begin{enumerate}
	\item $v (0) = 0$,
	\item $v (1) = 1$,
	\item $v (x \cdot y) = v (x) \cdot v (y)$ for all $x, y \in R$, and
	\item $v (x + y) \le \max {v (x), v (y)}$ for all $x, y \in R$.
\end{enumerate}

For instance, when $R$ is a Dedekind domain of Krull dimension 1 with fraction field $K$ and $\m$ is a maximal ideal of $R$, we can define a function $c_\m : R \to \ZZ \cup \{ \infty\}$ by sending zero to $\infty$ and any nonzero $r \in R$ to the number of times that $\m$ appears in the factorization of the principal ideal $(r)$. Note that this number is well defined because ideals in a Dedekind domain admit a unique factorization into maximal ideals, up to reordering. We can extend the function $c_\m$ to $K$ by setting $c_\m (r/s) := c_\m (r) - c_\m (s)$ for any $r, s \in R$ with $s \ne 0$.

Then, for any real number $n_\m > 1$, it is easy to check that the function
$ v_\m: K \to n_\m^{\mathbb{Z} \cup \{-\infty\}} = n_\m^\mathbb{Z} \cup \{0\}$ sending $x$ to $n_\m^{-c_\m(x)}$ is a valuation on $K$, which we call the \textit{$\m$-adic valuation}. For example, when $R = \ZZ$, $K = \QQ$ and $\m = (p)$ is the ideal generated by a prime number $p$, $v_p$ is the $p$-adic (multiplicative) valuation on $\QQ$, also known as the $p$-adic absolute value.

Note that the choice of number $n_\m > 1$ in the definition of $v_\m$ is not relevant, in the sense that any two such choices will give rise to equivalent valuations. In our formalization, we abstract the codomain $n_\m^\mathbb{Z} \cup \{0\}$ using the type
\texttt{with\_zero (multiplicative $\mathbb{Z}$)}, denoted $\ZZ_{m0}$,
which carries the multiplicative structure corresponding to the additive structure on $\ZZ$, as well as an added zero element.

For example, the rational numbers $20$ and $25$ have $5$-adic valuation $(-1 : \ZZ_{m0})$ and $(-2 : \ZZ_{m0})$, respectively, so $25$ is smaller than $20$ in the $p$-adic topology. \mi{Do you want to add anything else about this?}

We represent valuations in Lean using the \href{https://leanprover-community.github.io/mathlib_docs/ring_theory/valuation/basic.html#valuation}{valuation} structure available in mathlib, which encodes our above definition. A valuation on a ring $R$ induces a topology on $R$ for which the ring operations are continuous, hence turning $R$ into a topological ring.

Mathlib also provides the class \href{https://leanprover-community.github.io/mathlib_docs/topology/algebra/valuation.html#valued}{valued}, which bundles together a ring with a
uniform space structure and a distinguished valuation that induces it. This class is designed for rings in which there is a canonical valuation. For example, on the field $\QQ_p$ of $p$-adic numbers, there is a unique nontrivial valuation, which we register as an instance of the valued class. By contrast, on the rational numbers there are infinitely many valuations, none of them preferred over the others, so we do not put a global valued instance on $\QQ$.

\mi{I would like to move this paragraph to section 2.2, after  valuations in DVRs have been discussed }. When implementing valuations on a discrete valuation ring $R$ with field of fractions $K$, we made the design choice to put an instance of the valued class on $K$, but not on the ring $R$ itself. The reason is that, if needed, we can recover the
topological structure on $R$ from the one on $K$, so we prefer not to duplicate this information. \mi{This choice is consistent with mathlib's file \url{ring_theory.dedekind_domain.adic_valuation}, in which valued instances are only put on fields (should we mention this?)}

Given a ring $R$ with a valuation $v$, the \textit{unit ball} \mi {or  \textit{ring of integers}} of $R$ is defined as the subring $R_0$ of elements with valuation less than or equal to one. In mathlib, this subring is denoted \href{https://leanprover-community.github.io/mathlib_docs/ring_theory/valuation/integers.html#valuation.integer}{valuation.integer}. If $R$ is a field, then $R_0$ is a \textit{valuation subring}, that is, a subring such that for every element $r$ of $R$, either $r \in R_0$ or $r^{-1} \in R_0$.
To take advantage of the results about valuation subrings available in \href{https://leanprover-community.github.io/mathlib_docs/ring_theory/valuation/valuation_subring.html#valuation_subring}{mathlib}, when $R$ is a field we use the definition \href{https://leanprover-community.github.io/mathlib_docs/ring_theory/valuation/valuation_subring.html#valuation.valuation_subring}{valuation.valuation\_subring} to define the unit ball. This should not be confused with mathlib's \href{https://leanprover-community.github.io/mathlib_docs/ring_theory/valuation/valuation_ring.html#valuation_ring}{valuation\_ring} class.
 
\mi{
\begin{itemize}
	\item Properties of valuations valued in $\ZZ_{m0}$, together with the \code{multiplicative} and \texttt{with\_zero} classes, and clashes with "usual" intuition. Keep in mind that \texttt{with\_zero multiplicative} $\ZZ$ is "the same" as $q^{-\ZZ}\cup \{0\}$.
	\item Difference between \texttt{valued} (which is a class, and carries a topology/uniformity) and \texttt{valuation} that is just a function.
	\item We put \texttt{valued} instances on fields but "never" on rings that are not fields.
	\item There is something called \texttt{valuation\_ring} that has (almost) nothing to do; but \texttt{valuation\_subring} has.
	\item Comment about the choice of \texttt{.integers} instead of \texttt{.valuation\_subring}
\end{itemize}
}
\subsection{Definition of Discrete Valuation Rings}\label{subsection:def_dvr}
Let $R$ be a ring, \fil{as above assumed} commutative and unitary. We begin by recalling a classical result:
\begin{theorem}[see\dots]\label{thm:dvr_tfae}
The following properties are equivalent:
\begin{listResults}
\item foo
\item foo
\item foo
\end{listResults}
\end{theorem}
A ring satisfying the equivalent properties of Theorem~\ref{thm:dvr_tfae} is called a discrete valuation ring (often shortened as DVR). The definition is motivated by the fact that a DVR $R$ can always be endowed with a class of valuations, 
\[
v\colon R\longrightarrow \ZZ\cup\{+\infty\}\fil{\text{ or use the mathlib convention?}}.
\]
For each of these valuations, the image $\image(v)\subseteq\ZZ\cup \{+\infty\}$ is the free monoid generated by $v(r)$, where $r$ is any generator of the maximal ideal $\maxid$ of $R$. Upon replacing $v$ by $v/v(r)$, for some generator $r$ of $\maxid$, we can assume that this free monoid coincides with $\NN$: in this case the valuation is said to be normalised, and the elements of valuation equal to $1$ are called uniformizers. One basic result is that, for a normalised valuation, the uniformisers are exactly the set of generators of $\maxid$, \fil{because \dots}. Moreover, there exists a unique normalised valuation on a DVR, since every valuation is uniquely determined by its value on the generators of $\maxid$. We denote it by ${\normalised}_R$, or simply by $\normalised$ when the DVR $R$ can be understood.

This point of view differs slightly from the classical one (as found, for instance, in~\cite[Chapitre~I]{Ser62} or in~\cite[Chapitre~VII]{Bou07}), who only consider the normalised valuation on a DVR, and regard it as being ``canonical''. This should not come as a surprise in the setting of a formalisation work: on the one hand, regarding ``the'' valuation as part of the data would \fil{rigidify the situation too much NOT NICE}, as in \fil{give a REF to some discussion in our DD paper about fraction fields} and, on the other, there might be different constructions of a valuation on a ring that would turn out to be propositionally equal albeit not definitionally equal. The corresponding pairs $(R,v)$, for the several possible constructions of $v$, would all be mathematically intercheangable yet represented by different, although equivalent types; this would lead to unnecessary verbosity while keeping track of the equivalences: in \fil{\S\dots and in \S\dots we will see explicit examples of this phenomenon, supporting our decision of avoiding the choice of a preferred valuation on $R$}. In particular, the definition we propose of a discrete valuation is the following:
\begin{lstlisting}
definition (K : Type*) [field K] : is_discrete
\end{lstlisting}
This \fil{differs} from the usual definition (see, for instance~\cite[Proposition~I.1]{Ser62}), where one declares a valuation to be discrete when its image is isomorphic to a subgroup of $\ZZ$ (enriched with an element $\infty$), and observes that such a valuation can always be normalized to become surjective. In particular, with our definition, a valuation is discrete if and only if there exists a uniformizer, which we prove in the following form:
\begin{lstlisting}
lemma is_discrete_of_exists_uniformizer {K : Type*} [field K] (v : valuation K \ZZ_{m0}) {$\pi$ : K} (h$\pi$ : is_uniformizer v $\pi$) : is_discrete v := sorry
\end{lstlisting}
together with the lemma \code{is_discrete_of_exists_uniformizer} proving the reverse implication. Similarly, the aforementioned correspondance between uniformizers for a normalised valuation and generators of $\maxid$ takes the form
\begin{lstlisting}
lemma uniformizer_is_generator ($\pi$ : uniformizer v) : maximal_ideal v.valuation_subring = ideal.span {$\pi.1$} := sorry
\end{lstlisting}
where the fact that the valuation is normalised becomes \emph{a consequence}, rather than an assumption as in the classical approach. For the same reason, in the declaration formalising the opposite implication, namely
\begin{lstlisting}
lemma is_uniformizer_of_generator {r : K$_0$} [v.is_discrete]
  (hr : maximal_ideal v.valuation_subring = ideal.span {r}) : is_uniformizer v r :=
\end{lstlisting}
the assumption that \code{v} is discrete --- implemented as the \code{class} assumption \code{[v.is_discrete]} --- becomes necessary.


Since a DVR is an integral domain, it is possible to extend the valuation to its field of fractions $K=\Frac(R)$ by setting $v(a/b)=v(b)-v(a)$ \fil{or...}. If the valuation $v$ on $R$ is normalised, we obtain a \emph{surjective} valuation
\[
K\longtwoheadrightarrow \ZZ\cup\{+\infty\}\text{\fil{or\dots}}
\]
whose kernel coincides with $\unit{R}$. \fil{With the definitions introduced in} \S\ref{subsection:valuations}, this turns $K$ into a discretely valued field. 


Both DVRs and valuations had already been defined in \fil{\mathlib SAY WHAT THIS IS SOMEWERE} before our project, but nothing beyong their basic properties was available. In particular, no definition of uniformisers or the relation between generators of $\maxid$ and the valuation on a DVR was available. More crucially, two fundamental constructions were not formalised in \mathlib nor, \fil{to the best of our knowledge, in any other proof assistant CHECK}: extensions and completions of DVR. The next section is devoted to a description of our formalisation of these \fil{notions? constructions?}.
\begin{itemize}
	\item True Math definition and the \texttt{mathlib\_tfae} (in particular, local and PID)
	\item \texttt{is\_discrete} : observe that if the value group were $2\ZZ$ this would still be mathematically "discrete", but we make the choice of surjectivity in coherence with Serre/so that the notion of equivalence collapses to equality: there is a unique normalized one.
	\item \texttt{uniformizer}'s and equivalence of their existence with discreteness
	\fil{\item The unit ball of a field with a discrete valuation (be it valued or not) is a DVR; and the adic valuation (ref to María Inés' paper) on the field of fractions of a DVR \texttt{is\_discrete}}.
\end{itemize}
\subsection{Complete Discrete Valuation Rings and their Extensions}\label{subsec:complete_DVR}
Suppose now that $R$ is a DVR. \fil{As discussed in \S\ref{subsection:def_dvr}, $R$ is a local ring of Krull} dimension $1$ that is a principal ideal domain (shortened as PID). Its unique maximal ideal $\maxid$ is a height-one prime, and one can consider the adic valuation associated to this prime, \fil{as defined, for instance, in\dots}. In fact, it turns out that since $R$ is a PID, it is in particular (\fil{see\dots}) a Dedekind domain: now, Baanen~\emph{et al.}~have formalised in~\cite{BaaDahNarNuc21} and~\cite{BaaDahNarNuc22} the general theory of Dedekind domains in \lean, and de~Frutos-Fernández \fil{is this the right paper?} has formalised in \lean the main properties of adic valuations on Dedekind domains and on their extensions. The main result connecting the two theories is that the $\maxid$-adic valuation $v_{\maxid}$ associated to the prime ideal $\maxid$ coincides with the normalised valuation $\normalised$ on $R$. Although this is mathematically trivial, and the two functions
\[
\normalised,v_{\maxid}\colon R\longrightarrow \ZZ\cup\{+\infty\}
\]
are considered identical in pen-and-paper mathematics, they actually belong to different types and hence are different terms. \fil{Indeed\dots WAIT FOR A THROUGHOUT DISCUSSION OF ADIC VAL'S} 

\vspace{3cm}

A pivotal construction in this context --- upon which the main application of the theory of DVRs and of local fields to algebraic number theory relies --- is that of the \emph{completion} of a DVR, or of its field of fractions, with respect to an adic valuation.

 and we can regard therefore $\maxid$ is its unique non-zero prime ideal. Hencee can regard it as 
\begin{itemize}
	\item The completion of a discretely valued field is again discretely valued (and so its ring of integers is a DVR by the above)
	\fil{\item The equality of the two valuations (the adic one coming from the max'l ideal, since DVR implies local) and the uniform completion of the valuation we begun with
	\item When viewing $K_v$ as the completion of $K$, its \texttt{valued} instance comes from the completion of the valuation on $K$, and this is of course different from the \texttt{valued} instance on the fraction field of $R_v$, itself isomorphic to $K_v$, that instead comes from the \code{discrete_valuation_ring} instance on $R_v$.
	\item Still, the isomorphism is known and one can compare the two valuations on $K_v$ which is \emph{a fraction field}; this is done in \code{v_1=v_2}.}
	\mi{\item Recall (María Inés' paper): there is a norm associated to a valuation; and if the field below is complete and the extension is finite, then this norm is unique (extending the one below).
	\item Speak about the work to define the normalized \texttt{discrete} valuation "above"
	\item [MAYBE] Mention the lemma \texttt{w\_le\_one\_iff\_disc\_norm\_extension\_le\_one}
	\item The proof (from previous project) and instance that a finite extension of a complete field is again complete
	\item The lemma \code{integral_closure_eq_integer} and the theorem \code{dvr_of_finite_extension}
	\item The extension of the valuation is not a valued instance because this would create a diamond for every valued field K. But we do get a DVR instance on the integral closure of the integers.}
\end{itemize}
\section{Local Fields}
Give our definition and the claims that finite extensions of $\laurentseries$ and of $\QQ_p$ are local fields.\\
The valued instance on a mixed/equal characteristic local field causes a diamond on $\QQ_p$ or
$\laurentseries$.
\subsection{Equal characteristic}
\begin{itemize}
	\fil{\item Unlike the $p$-adic, the completion of the ring of rational functions had not yet been defined in mathlib.
	\item Put both a valuation and a norm on $\laurentseries$
	\item iso between Laurent series and the completion}
	\mi{\item Definition of Eq char local field
	\item Our definition of ring of integers is the integral closure of $\powerseries$ in $K$, mainly to adapt to what has been done for number fields.
	\item The definition \code{ring_of_integers_algebra}
	\item $\laurentseries$ is an eq. char. local ring [(and also $\laurentseries[{\FF[q]}]$?)}
	
\end{itemize}
\subsection{Mixed characteristic}
\begin{itemize}
	\mi{\item Definition of Mixed char local field
	\item Our definition of ring of integers is the integral closure of $\ZZ_p$ in $K$, mainly to adapt to what has been done for number fields.
	\item The definition \code{ring_of_integers_algebra}, "equal" to the one for eq. char. local fields}
	\fil{\item Discussion about the two $\QQ_p$'s from a mathematical point of view.
	\item The proof that $\QQ_p$ is a mixed char local field.
	\item Say that most of the techniques used for the iso about Laurent series adapted to \code{padic_ring_equiv} IF TRUE}
	\item A proof that the ring of integers is a DVR (once $\ZZ_p$ becomes the right $\ZZ_p$ --- so, not only a DVR, which it is, but the \code{valuation_subring} of $\QQ_p$ --- this will follow from the general results about DVR's).
\end{itemize}


\section{Discussion}
\begin{itemize}
	\item Localizations of number fields are local fields
	\item Future work (not too much, but towards LCFT)
	\item Related works
\end{itemize}
\subsection{Implementation Details}
\begin{itemize}
	\item \mi{bundled vs unbundled \texttt{uniformizer}'s and the choice of having them in K or in R or in R.int\dots}
	\item \fil{insert here the "double" $Q_p$ in particular the fact we found a \code{metric} instance on $\QQ$}.
	\item \mi{the loop forcing the instance \texttt{discretely\_normed\_field} to be only a local one
	\item [MAY BE] Discuss working in $\ZZ_{m0}$ vs in $\RR$ for the DVR\_extension
	\item The instance of \code{valued} at the end of basic is defined on THE fraction ring and not on A fraction field because otherwise this would get into a diamond. Look at the examples in \code{extension}. Observe that \code{height_one_spectrum.valued_adic_completion K v} takes as argument $K$ (that is A fraction ring of $R$), whereas \code{discrete_valuation.with_zero.valued} takes as argument a DVR (in this case, $R_v$). Since the completion at $v$ of the fraction ring $frac R$ is not equal to $K_v$, this leads to no problem.}
\end{itemize}

%%
%% The acknowledgments section is defined using the "acks" environment
%% (and NOT an unnumbered section). This ensures the proper
%% identification of the section in the article metadata, and the
%% consistent spelling of the heading.
%TODO We should add the funding ack. here (this is commented out by the `anonymous` option).
\begin{acks}
\mi{}
\end{acks}

%%
%% The next two lines define the bibliography style to be used, and
%% the bibliography file.
\bibliographystyle{ACM-Reference-Format}
\bibliography{CPP_biblio}


%%
%% If your work has an appendix, this is the place to put it.
%\appendix


\end{document}
\endinput
%%
%% End of file `sample-sigplan.tex'.
