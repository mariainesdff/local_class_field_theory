\documentclass[11pt,a4paper]{amsart}

\title{Local Class Field Theory}




%%==============================================================================
%
%              PACKAGES
%
%%==============================================================================


\usepackage[english]{babel}
\usepackage{enumerate}
\usepackage{setspace}
\usepackage[project=..]{blueprint}
\usepackage{amsfonts,amsthm,amsmath,amssymb,comment,url}

% Set up blueprint options


%\home{https://leanprover-community.github.io/flt-regular/}
%\github{https://github.com/leanprover-community/flt-regular/}


%\usepackage[ocgcolorlinks]{hyperref}
%\hypersetup{citecolor=blue,
%	linkcolor=red}



%%==============================================================================
%
%              ADDRESSES
%
%%==============================================================================
\author{María Inés de Frutos Fernández}
\author{Filippo A.~E.~Nuccio Mortarino Majno di Capriglio}	
%\address{Department of Mathematics, University College London, Gower street, London, WC1E 6BT}
%\email{c.birkbeck@ucl.ac.uk} 
%\urladdr{orcid.org/0000-0002-7546-9028}



%%==============================================================================
%
%              COMMENTS
%
%%==============================================================================



%%==============================================================================
%
%              THEOREM STYLES
%
%%==============================================================================
\theoremstyle{plain}
\newtheorem{theorem}[section]{Theorem}
\newtheorem*{theorem*}{Theorem}
\newtheorem{lemma}[section]{Lemma}
\newtheorem{corollary}[section]{Corollary}
\newtheorem{conjeture}[section]{Conjecture}
\newtheorem*{conjecture*}{Conjecture}
\newtheorem{proposition}[section]{Proposition}
\newtheorem*{proposition*}{Proposition}
\theoremstyle{definition}
\newtheorem{definition}[section]{Definition}
\newtheorem{nota}[subsection]{Notation}
\theoremstyle{remark}
\newtheorem{example}[section]{Example}
\newtheorem{remark}[section]{Remark}
\newtheorem*{remark*}{Remark}


%%==============================================================================
%
%              MACROS
%
%%==============================================================================

%
%
%              END PREAMBLE
%
%%==============================================================================
\begin{document}
\documentclass[11pt]{amsart}
\usepackage{amsfonts, amsthm, amssymb, amsmath}
\usepackage{tikz-cd}
\usetikzlibrary{decorations}

\usepackage{unicode-math}
%\usepackage{fontspec}
\setmathfont{latinmodern-math.otf}
%\setmathfont[range=\varnothing]{Asana-Math.otf}


\usepackage{blueprint}

\setlength{\textwidth}{6.5in}
\setlength{\oddsidemargin}{-0.1in}
\setlength{\evensidemargin}{-0.1in}

%\numberwithin{equation}{subsection}
%\newtheorem{theorem}{Theorem}
%\numberwithin{theorem}{subsection}
%\newtheorem*{theoremx}{Theorem} % theorem not in dep graph
%\newtheorem*{remarkx}{Remark} % unnumbered remark
%\newtheorem{lemma}[theorem]{Lemma}
%\newtheorem{lemmax}[theorem]{Lemma} % lemma not in dep graph
%\newtheorem{corollary}[theorem]{Corollary}
%\newtheorem{proposition}[theorem]{Proposition}
%\newtheorem{conjecture}[theorem]{Conjecture}
%\newtheorem{problem}[theorem]{Problem}
%\newtheorem{assumption}[theorem]{Assumption}
%\newtheorem{defprop}[theorem]{Definition/Proposition}

%\newcommand*{\unitball}[1]{{#1}^\circ}
\newcommand*{\ZZ}{\mathbb{Z}}
\newcommand*{\Qp}[1][p]{\mathbb{Q}_{#1}}
\newcommand*{\Fp}[1][p]{\mathbb{F}_{#1}}
\newcommand*{\laurentseries}[1][\Fp]{{#1}(\!(X)\!)}

%\theoremstyle{definition}
%\newtheorem{remark}[theorem]{Remark}
%\newtheorem{exercise}[theorem]{Exercise}
%\newtheorem{warning}[theorem]{Warning}
\newtheorem{definition}{Definition}
%\newtheorem{question}[theorem]{Question}
%\newtheorem{example}[theorem]{Example}
%\newtheorem{examples}[theorem]{Examples}
%\newtheorem{observation}[theorem]{Observation}


\date{\today}

\setcounter{tocdepth}{1}

\title{Blueprint for the LCFT}

\author{MIdFF and FAEN}

\begin{document}
\section{Intro}
\begin{definition}
\label{def_local_field}
\lean{foo}
A local field is a field
\end{definition}

%\bibliographystyle{alpha}
%
%\bibliography{Analytic}

\end{document}
	
	
%	\bibliographystyle{alpha}
%	
%	
%	\bibliography{bibliog}
	
	
	
\end{document}	
