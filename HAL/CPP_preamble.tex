\usepackage[T1]{fontenc}
% \PassOptionsToPackage{rgb, dvipsnames}{xcolor}
\definecolor{Burgundy}{RGB}{128,0,32}
\usepackage{enumitem}
\usepackage{amsfonts, amsthm, amsmath}
\usepackage{soul}
\usepackage[english]{babel}
\hyphenation{non-archi-me-de-an}
% \usepackage{dutchcal}
\usepackage{mathrsfs}
\usepackage{fontawesome} % external link icon
\usepackage{xpunctuate}
\usepackage{hyperref}
\usepackage{microtype}
 \usepackage{balance}
%\hypersetup{%
%	colorlinks=true,
%	linkcolor=blue,
%	filecolor=magenta,      
%	urlcolor=blue,
%}
%%%%%%%%%%%%%%%%%%%%%%%%%%%%%%%%%%%%%%%%%%%
% Typography
%%%%%%%%%%%%%%%%%%%%%%%%%%%%%%%%%%%%%%%%%%%
% \newcommand{\fil}[1]{{\color{blue} #1}}
% \newcommand{\mi}[1]{{\color{purple} #1}}
%\DeclareRobustCommand{\extlink}[1]{\href{{#1}}{\ensuremath{{}^\text{\faExternalLink}}}}
\newcommand{\extlink}{~\ensuremath{{}^\text{\faExternalLink}}}
\newcommand*{\ibid}{\emph{ibid}\xperiod}
%%%%%%%%%%%%%%%%%%%%%%%%%
%%Lean-related typography
%%%%%%%%%%%%%%%%%%%%%%
\usepackage{listings}
\def\lstlanguagefiles{lstleanCPP.sty}
\lstset{language=lean}
%\definecolor{keywordcolor}{rgb}{0.7, 0.1, 0.1}   % red
%\definecolor{commentcolor}{rgb}{0.4, 0.4, 0.4}   % grey
%\definecolor{symbolcolor}{rgb}{0.4, 0.4, 0.4}    % grey
\renewcommand{\lstlistingname}{Code excerpt}
\newcommand{\LClistingname}{\MakeLowercase{\lstlistingname}}
\newcommand{\code}[1]{\lstinline{#1}}
\newcommand{\mcode}[1]{\mathtt{\ #1\ }}
\newcommand*{\lean}[1][3]{Lean~{#1}\xspace}
\newcommand*{\mathlib}{\texttt{mathlib}\xspace}
% \newcommand{\pfomit}{$ - - QED$}
%%%%%%%%%%%%%%%%%%%%%%%%%%%%%%%%%%%%%%%%%%%
%%%Theorems
%%%%%%%%%%%%%%%%%%%%%%%%%%%%%%%%%%%%%%%%%%%
\theoremstyle{plain}
\newtheorem{theorem}{Theorem}[section]
\newtheorem{proposition}[theorem]{Proposition}
\newtheorem{corollary}[theorem]{Corollary}
\theoremstyle{definition}
\newtheorem{definition}[theorem]{Definition}
\newtheorem{notation}[theorem]{Notation}
%
\theoremstyle{remark}
\newtheorem{remark}[theorem]{Remark}
%%%%%%%%%%%%%%%%%%%%%%%%%%%%%%%%%%%%%%%%%%%
%%%%%Lists
%%%%%%%%%%%%%%%%%%%%%%%%%%%%%%%%%%%%%%%%%%%
\newlist{listResults}{enumerate}{1}
\setlist[listResults]{label={\arabic*}.}
\newlist{sublistResults}{enumerate}{1}
\setlist[sublistResults]{label={\arabic{listResultsi}}-\,\alph*\textup{)}}
\newlist{listConditions}{enumerate}{1}
\setlist[listConditions]{label=\textup{({\Alph*})}}
\newlist{listDef}{enumerate}{1}
\setlist[listDef]{label={\roman*}\textup{)}}
\newlist{listHyp}{enumerate}{1}
%%%%%%%%%%%%%%%%%%%%%%%%%%%%%%%%%%%%%%%%%%%
% Math commands
%%%%%%%%%%%%%%%%%%%%%%%%%%%%%%%%%%%%%%%%%%%
%%Fields
\newcommand*{\NN}{\mathbb{N}}
\newcommand*{\ZZ}{\mathbb{Z}}
\newcommand*{\wzmZZ}{\ZZ_{\mathtt{m}0}} %% With-zero multiplicative version of \ZZ
\newcommand*{\wzmNN}{\NN_{\mathtt{m}0}} %% With-zero multiplicative version of \NN
\newcommand*{\RR}{\mathbb{R}}
\newcommand*{\QQ}{\mathbb{Q}}
\newcommand*{\CC}{\mathbb{C}}
\newcommand*{\FF}[1][p]{\mathbb{F}_{#1}}
\newcommand*{\m}{\mathfrak{m}}
\newcommand*{\av}[1][\empty]{a_{#1}}
\DeclareMathOperator{\Frac}{Frac} %the fraction field
%%Algebraic
\newcommand*{\laurentseries}[1][{\FF}]{\ensuremath{#1(\!(X)\!)}} %Laurent series
\newcommand*{\puiseuxseries}[1][{R}]{\ensuremath{#1\{\!\{X\}\!\}}} %Laurent series
\newcommand*{\compl}[2][\empty]{\ensuremath{\widehat{#2}_{#1}}} %General completion
\newcommand*{\FpXCompl}[1][{\FF}]{\compl{#1(X)}}  %Adic completion of Rational functions
\newcommand*{\FpXComplInt}[1][\FF]{\ball{\bigl(\FpXCompl[#1]\bigr)}}  %Adic completion of polynomials
\newcommand*{\padicCompl}{\compl[(p)]{\QQ}}  %Adic completion of Rational numbers
\newcommand*{\padicComplInt}{\ball{\bigl(\padicCompl\bigr)}}  %Adic completion of the integers
\newcommand*{\powerseries}[1][{\FF}]{#1{[\![X]\!]}} %Power series
\newcommand*{\maxid}[1][R]{\mathfrak{m}_{#1}} %Maximal ideal (of R)
\newcommand{\unit}[1]{{#1}^\times} % units
%% Arithmetic
\newcommand*{\rint}[1]{\mathcal{O}_{#1}} % ring of integers
%% Valuations
\newcommand*{\normalised}[1][v]{{#1}_{0}} %The normalised valuation attached to v
\newcommand*{\ball}[1]{{#1}^\circ} % ring of integers
%%General set-theoretic
\newcommand*{\filter}{\mathscr{F}}
\newcommand*{\princfilter}{\boldsymbol{\mathcal{P}}}
\newcommand*{\nhds}{\boldsymbol{\mathcal{N}}}
\DeclareMathOperator{\image}{Im} %image of a map
\newcommand*{\longtwoheadrightarrow}{\joinrel\relbar\joinrel\twoheadrightarrow}%Long surjective arrow